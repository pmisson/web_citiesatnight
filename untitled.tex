
\begin{figure}
\includegraphics[width=15cm]{Figuras/Fuentes/Colors_and_extinction.png} 
\begin{minipage}[]{14 cm}
\captionof{figure}{Los puntos corresponden a diferentes extinciones y alturas. El punto m\'as cercano a la esquina superior izquierda corresponde a el color en laboratorio, el siguiente a $k_B=0.3$,$k_V=0.15$ y $k_R=0.08$ (alias baja extinci\'on) con inclinaci\'on de 0 grados, el siguiente igual extinci\'on pero inclinaci\'on de 60 grados. Los siguientes con extinciones de $k_B=0.81$,$k_V=0.41$ y $k_R=0.2$ (alias alta extinci\'on) a inclinaciones de 0 grados y 60 grados. Los puntos corresponden a los mostrados en la tabla \ref{coloresVIIRS}. Se observa que salvo para el caso extremo de alta extinci\'on y alta inclinaci\'on, los colores de las fuentes se mantienen en el entorno marcado por las l\'ineas rojas que delimitan las regiones que clasificamos las l\'amparas en la secci\'on \ref{Madrid_seleccion}   \label {colorextincion}}
\label{fig:examplefl}


\begin{tabular}{|r||r|r|r|r||r|r|r|r||r|r|r|r||}
\hline
\multicolumn{1}{|l||}{Banda} & \multicolumn{4}{c||}{B}  & \multicolumn{4}{c||}{V} & \multicolumn{4}{c||}{R}  \\ \hline
\multicolumn{1}{|l||}{$\delta\theta$ } & 0.3 & 0.4 & 0.63 & 0.81 & 0.15 & 0.2 & 0.32 & 0.41 & 0.08 & 0.1 & 0.16 & 0.2 \\ \hline
0 & 0.76 & 0.69 & 0.56 & 0.47 & 0.87 & 0.83 & 0.75 & 0.69 & 0.93 & 0.91 & 0.86 & 0.83 \\ \hline
10 & 0.76 & 0.69 & 0.55 & 0.47 & 0.87 & 0.83 & 0.74 & 0.68 & 0.93 & 0.91 & 0.86 & 0.83 \\ \hline
20 & 0.75 & 0.68 & 0.54 & 0.45 & 0.86 & 0.82 & 0.73 & 0.67 & 0.93 & 0.91 & 0.86 & 0.82 \\ \hline
30 & 0.73 & 0.65 & 0.51 & 0.42 & 0.85 & 0.81 & 0.71 & 0.65 & 0.92 & 0.9 & 0.84 & 0.81 \\ \hline
40 & 0.7 & 0.62 & 0.47 & 0.38 & 0.83 & 0.79 & 0.68 & 0.61 & 0.91 & 0.89 & 0.83 & 0.78 \\ \hline
50 & 0.65 & 0.56 & 0.4 & 0.31 & 0.81 & 0.75 & 0.63 & 0.56 & 0.9 & 0.87 & 0.8 & 0.75 \\ \hline
60 & 0.58 & 0.48 & 0.31 & 0.22 & 0.76 & 0.69 & 0.56 & 0.47 & 0.87 & 0.83 & 0.75 & 0.69 \\ \hline
\end{tabular}
\captionof{table}{Transmisi\'on de la atm\'osfera para varios valores de la extinci\'on en la bandas B, 
V y R. La extinci\'on est\'a expresada en magnitudes.}
\label{}

\end{figure} 